\documentclass{article}

\usepackage{./prelude}

\DeclareMathOperator{\mesh}{mesh}

\Class{Math 424}
\Title{Final Theorems}
\Author{Your Name}{netid}

\begin{document}

\begin{problem}
Let $s$ be a Cauchy sequence in $E$.
Prove that if a subsequence of $s$ converges to $L \in E$, then the whole sequence converges to $L$.
\begin{solution}
	Let $\eps > 0$ and $\set{s_{n_i}}$ denote the subsequence that converges to $L$.
	Since $\set{s_n}$ is Cauchy, there is an $N \in \N$ so that for all $n, m > N$, we have $d(s_n, s_m) < \frac\eps2$.
	Moreover, since $\set{s_{n_i}}$ converges to $L$, there is an $M \in \N$ so that for all $i \ge M$, we have $d(s_{n_i}, L) < \frac\eps2$.
	Thus, for all $n > K \coloneq \max(N, M)$, we have
	\[
		d(s_n, L) \le d(s_n, s_{n_K}) + d(s_{n_K}, L) < \frac\eps2 + \frac\eps2 = \eps,
	\]
	i.e. $s_n$ converges to $L$.
\end{solution}
\end{problem}

\begin{problem}
Prove that in any metric space, compact sets are closed.
\begin{solution}
	Let $K$ be compact.
	We prove that $K$ is closed by showing that $K^C$ is open.

	Let $p \in K^C$ and consider the collection of nested open sets $O \coloneq \set{ U_n \coloneq \overline{B\left(p, \frac1n\right)}^C : n \in \N }$.
	Observe that $\bigcup U_n \supset E \setminus \set{ p } \supset K$.
	Thus, since $K$ is compact, we obtain a finite subcover $U_{n_1}, \dots, U_{n_k}$ of $K$.
	Without loss of generality, assume that $n_1 > \dots > n_k$.
	Then, since $U_1 \subset U_2 \subset \cdots$, we have $\bigcup U_{n_i} = U_{n_1} \supset K$.
	Hence, $B\left(p, \frac1{n_1}\right) \subset \overline{B\left(p, \frac1{n_1}\right)} = U_{n_1}^C \subset K^C$, as desired.
\end{solution}
\end{problem}

\begin{problem}
Prove that in any metric space, compact sets are sequentially compact.
\begin{solution}
	Let $K$ be compact and suppose for contradiction that $K$ is not sequentially compact.
	Then, there is a sequence $\set{ s_n } \subset K$ that does not have a convergent subsequence.
	Equivalently, $\set{ s_n }$ has no cluster points, so for every $x \in K$, there is an $\eps_x > 0$ so that $B(x, \eps_x)$ contains only finitely many sequence members $\set{ s_n }$.
	We therefore construct the open cover $O \coloneq \set{ U_x \coloneq B(x, \eps_x) : x \in K }$ of $K$.
	Since $K$ is compact, we obtain a finite subcover $U_{x_1}, \dots, U_{x_k}$ of $K$.
	But since we have finitely many $U_{x_i}$ that each contain only finitely many $s_n$, the union $\bigcup U_{x_i}$ contains only finitely many sequence members $s_n$, a contradiction since $\bigcup U_{x_i} \supset K$ contains infinitely many!
\end{solution}
\end{problem}

\begin{problem}
Prove that if $f_n : E \to E'$ are continuous and $f_n$ converges to $f : E \to E'$ uniformly on $E$, then $f$ is continuous.
\begin{solution}
	Let $\eps > 0$ and fix $p \in E$.
	Since $f_n \to f$ uniformly, there is an $N \in \N$ so that $d(f_N(x), f(x)) < \frac\eps3$ for all $x \in E$.
	Moreover, since $f_N$ is continuous, there is a $\delta > 0$ so that, for every $x \in E$ with $d(x, p) < \delta$, we have $d(f_N(x), f_N(p)) < \frac\eps3$.
	Hence,
	\[
		d(f(x), f(p)) \le d(f(x), f_N(x)) + d(f_N(x), f_N(p)) + d(f_N(p), f(p)) < \frac\eps3 + \frac\eps3 + \frac\eps3 = \eps,
	\]
	i.e. $f$ is continuous at $p$.
\end{solution}
\end{problem}

\begin{problem}
State and prove Rolle's Theorem and the Mean Value Theorem.
\begin{solution}
	\begin{theorem*}[Rolle]
		Let $f : [a, b] \to \R$ be continuous on $[a, b]$, differentiable on $(a, b)$, and have $f(a) = f(b)$.
		Then, there is some $c \in (a, b)$ so that $f'(c) = 0$.
	\end{theorem*}
	\begin{proof}
		Since $f$ is continuous on $[a, b]$ compact, it attains its max and min, say $M$ and $m$ respectively.
		If $M = m = f(a) = f(b)$, then $f$ must be constant of $[a, b]$.
		Hence, any $c \in (a, b)$ satisfies $f'(c) = 0$.
		Otherwise, one of $M$ or $m$ (say $M$ for definiteness) is not equal to $f(a)$.
		We therefore have a $c \in (a, b)$ so that $f(c) = M$, which implies $f'(c) = 0$ by the Max-Min Test.
	\end{proof}

	\begin{theorem*}{Mean Value Theorem}
		Let $f : [a, b] \to \R$ be continuous on $[a, b]$ and differentiable on $(a, b)$.
		Then, there is some $c \in (a, b)$ so that $f'(c) = \frac{f(b) - f(a)}{b - a}$.
	\end{theorem*}
	\begin{proof}
		Define the map $g : [a, b] \to \R$ by
		\[
			g(x) = f(x) - \frac{f(b) - f(a)}{b - a} (x - a).
		\]
		By continuity rules, $g$ is continuous on $[a, b]$.
		Similarly, by differentiability rules, $g$ is differentiable on $(a, b)$ with derivative $g'(x) = f'(x) - \frac{f(b) - f(a)}{b - a}$.

		Observe, too, that $g(a) = f(a)$ and $g(b) = f(b) - \frac{f(b) - f(a)}{b - a}(b - a) = f(a)$.
		Thus, by Rolle's Theorem, there is some $c \in (a, b)$ so that $g'(c) = f'(c) - \frac{f(b) - f(a)}{b - a} = 0$, i.e. $f'(c) = \frac{f(b) - f(a)}{b - a}$, as desired.
	\end{proof}
\end{solution}
\end{problem}

\begin{problem}
State and prove F.T.C. 1.
\begin{solution}
	\begin{theorem*}[F.T.C. 1]
		Let $f : [a, b] \to \R$ be continuous on $[a, b]$ and differentiable on $(a, b)$.
		If $f'$ is bounded and integrable on $[a, b]$, then $\int_a^b f'(x) \dd{x} = f(b) - f(a)$.
	\end{theorem*}
	\begin{proof}
		Fix a partition $P \coloneq t_0 < t_1 < \dots < t_n$ and apply the Mean Value Theorem on each interval of the partition:
		for every $1 \le k \le n$, there is an $x_k \in [t_{k - 1}, t_k]$ such that
		\[ m(f', [t_{k - 1}, t_k]) \le f'(x_k) = \frac{f(t_k) - f(t_{k - 1})}{t_k - t_{k - 1}} \le M(f', [t_{k - 1}, t_k]). \]
		Thus,
		\[
			L(f', P) \le \sum f(t_k) - f(t_{k - 1}) \le U(f', P)
		\]
		for every partition $P$, so $\int_a^b f' = L(f') \le f(b) - f(a) \le U(f') = \int_a^b f'$, i.e. $\int_a^b f' = f(b) - f(a)$.
	\end{proof}
\end{solution}
\end{problem}

\begin{problem}
State and prove F.T.C. 2.
\begin{solution}
	\begin{theorem*}[F.T.C. 2]
		Let $f$ be bounded and integrable on $[a, b]$.
		Then, $F(x) = \int_a^x f(t) \dd{t}$ is continuous on $[a, b]$.
		Moreover, if $f$ is continuous at $x_0$, then $F$ is differentiable at $x_0$ with derivative
		$F'(x_0) = f(x_0)$.
	\end{theorem*}
	\begin{proof}
		Since $f$ is bounded on $[a, b]$, there is an $M \ge 0$ so that $\abs{f(x)} \le M$ for all $x \in [a, b]$.
		Thus,
		\[
			\abs{F(x) - F(y)} = \abs{\int_y^k f(t) \dd{t}} \le \abs{\int_y^x \abs{f(t)} \dd{t}} \le \abs{\int_y^x M \dd{t}} = M \abs{y - x},
		\]
		i.e. $F$ is Lipschitz (and thus continuous) on $[a, b]$.

		Now, suppose that $f$ is continuous at $x_0$ and let $\eps > 0$.
		Then, there is a $\delta > 0$ so that for $x \in (x_0 - \delta, x_0 + \delta)$, we have $\abs{f(x) - f(x_0)} < \eps$.
		Thus,
		\[
			\abs{\frac{F(x) - F(x_0)}{x - x_0} - f(x_0)} = \abs{\frac1{x - x_0} \int_{x_0}^x f(t) \dd{t} - f(x_0)} = \abs{\frac1{x - x_0} \int_{x_0}^x f(t) - f(x_0) \dd{t}}.
		\]
		Since $t$ is between $x$ and $x_0$ and $\abs{x - x_0} < \delta$, we have $\abs{t - x_0} < \delta$ and thus, $\abs{f(t) - f(x)} < \eps$.
		Hence,
		\[
			\abs{\frac1{x - x_0} \int_{x_0}^x f(t) - f(x_0) \dd{t}} \le \abs{\frac1{x - x_0} \int_{x_0}^x \abs{f(t) - f(x_0)} \dd{t}} < \abs{\frac1{x - x_0} \int_{x_0}^x \eps \dd{t}} = \eps.
		\]
		Thus, the derivative quotient exists and is equal to $f(x_0)$.
	\end{proof}
\end{solution}
\end{problem}

\begin{problem}
State and prove the Cauchy criterion for Darboux integration.
\begin{solution}
	\begin{theorem*}[Cauchy Criterion]
		Let $f : [a, b] \to \R$ be bounded.
		Then, $f$ is integrable on $[a, b]$ if and only if, for all $\eps > 0$, there is a partition $P$ of $[a, b]$ so that $U(f, P) - L(f, P) < \eps$.
	\end{theorem*}
	\begin{proof}
		($\implies$).
		Suppose that $f$ is integrable on $[a, b]$.
		Then, $U(f) = L(f)$.
		Fix $\eps > 0$.
		Then, by the definition of $U(f) = \inf_P U(f, P)$, we have some partition $P$ so that $U(f, P) < U(f) + \frac\eps2$.
		Similarly, since $L(f) = \sup_P L(f, P)$, we have some partition $Q$ so that $L(f, P) > L(f) - \frac\eps2$.
		Thus, for the partition $R \coloneq P \cup Q$, we have
		\[
			U(f, R) - L(f, R) < U(f) - L(f) + \eps = \eps,
		\]
		as desired.

		($\impliedby$).
		Now, suppose that, for every $\eps > 0$, there is a partition $P$ so that $U(f, P) - L(f, P) < \eps$.
		Then,
		\[
			U(f) \le U(f, P) - L(f, P) + L(f, P) \le \eps + L(f, P) \le L(f) + \eps,
		\]
		so $U(f) - L(f) \le \eps$.
		Since this inequality holds for all $\eps > 0$, we thus conclude that $U(f) \le L(f)$ (and hence, $U(f) = L(f)$).
	\end{proof}
\end{solution}
\end{problem}

\begin{problem}
Prove that every continuous function $f : [0, 1] \to \R$ is Darboux integrable.
\begin{solution}
	Fix $\eps > 0$.
	Since $f$ is continuous on $[0, 1]$ compact, $f$ is uniformly continuous.
	Thus, there is some $\delta > 0$ so that for all $x, y \in [0, 1]$, if $\abs{x - y} < \delta$, then $\abs{f(x) - f(y)} < \eps$.
	Let $P$ be a partition of $[0, 1]$ with $\mesh(P) < \delta$.
	Then,
	\[
		U(f, P) - L(f, P) = \sum (M(f, [t_{i - 1}, t_i]) - m(f, [t_{i - 1}, t_i])) (t_i - t_{i - 1}).
	\]
	Since $\abs{t_i - t_{i - 1}} < \delta$, we must have $M - m < \eps$.
	Thus,
	\[
		U(f, P) - L(f, P) < \eps \sum t_i - t_{i - 1} = \eps,
	\]
	so $f$ is integrable by the Cauchy Criterion.
\end{solution}
\end{problem}

\newpage

\begin{problem}
Prove that if $f_n$ are bounded and integrable on $[0, 1]$ and $f_n$ converges to $f$ uniformly on $[0, 1]$, then $f$ is integrable on $[0, 1]$ and $\int_0^1 f_n \to \int_0^1 f$ as $n \to \infty$.
\begin{solution}
	Given $\eps > 0$, we can pick an $N \in \N$ so that, for all $n \ge N$, we have $\abs{f_n(x) - f(x)} < \eps$ for all $x \in [a, b]$ (since $f_n \to f$ uniformly).
	By the Cauchy Criterion, we can also pick a partition $P$ so that $U(f_N, P) - L(f_N, P) < \eps$.
	Now, note that
	\[
		U(f, P) \le U(f_N, P) + U(f - f_N, P) \le U(f_N, P) + \eps(b - a).
	\]
	Similarly,
	\[
		L(f, P) \ge L(f_N, P) - L(f - f_N, P) \ge L(f_N, P) - \eps(b - a).
	\]
	Thus,
	\[
		U(f, P) - L(f, P) \le U(f_N, P) - L(f_N, P) + 2\eps(b - a) < \eps + 2\eps(b - a).
	\]
	So, by the Cauchy Criterion, $f$ is integrable on $[a, b]$.

	Now, observe that
	\[
		\abs{\int_a^b f_n(x) \dd{x} - \int_a^b f(x) \dd{x}} \le \abs{\int_a^b \abs{f_n(x) - f(x)} \dd{x}} \le \eps(b - a).
	\]
	Thus, $\int_a^b f_n \to \int_a^b f$.
\end{solution}
\end{problem}

\begin{problem}
Prove that any power series with radius of convergence $R$ and center $x_0$ converges uniformly on any compact subset of the interval $(x_0 - R, x_0 + R)$.
\begin{solution}
	It suffices to prove that for any $0 \le R_1 < R$, the series converges on $[x_0 - R_1, x_0 + R_1]$.
	Recall that the series
	\[
		S \coloneq \sum_{n = 0}^\infty a_n (x - x_0)^n
	\]
	converges absolutely for $x \in (x_0 - R, x_0 + R)$.

	Thus, if $x \in [x_0 - R_1, x_0 + R_1] \subset (x_0 - R, x_0 + R)$, the series $\sum \abs{a_n} R_1^n$ converges, too.
	But since $\abs{a_n(x - x_0)^n} \le \abs{a_n} R_1^n$ here, $S$ converges uniformly on $[x_0 - R_1, x_0 + R_1]$ by the Weierstrass M-Test.
\end{solution}
\end{problem}

\begin{problem}
Prove Banach's Fixed Point Theorem:
Let $(E, d)$ be a complete metric space and let $T : E \to E$ be a contraction (i.e. $\exists c \in [0, 1)$, $\forall x, y \in E$, $d(T(x), T(y)) \le c d(x, y)$), then there exists a unique $p \in E$ so that $T(p) = p$.
\begin{solution}
	(Uniqueness).
	Suppose that $x, y \in E$ are fixed points of $T$.
	Then,
	\[
		0 \le d(x, y) = d(T(x), T(y)) \le c d(x, y).
	\]
	Since $c < 1$, we thus conclude that $d(x, y) = 0$, i.e. $x = y$.

	(Existence).
	Take $p_1 \in S$ and define a sequence by the recurrence $p_{n + 1} = T(p_n)$.
	Note that for $n \ge 2$, we have
	\[
		d(p_n, p_{n + 1}) = d(T(p_{n - 1}), T(p_n)) \le c^{n - 1} d(p_1, p_2).
	\]
	Thus, for $n < m$, we get
	\[
		d(p_n, p_m) \le d(p_n, p_{n + 1}) + \dots + d(p_{m - 1}, p_m) \le \sum_{k = n + 1}^{m} c^{k - 1} d(p_1, p_2) \le \sum_{k = n + 1}^\infty c^{k - 1} d(p_1, p_2).
	\]
	Since $c \in [0, 1)$, this geometric series converges, and thus, for any $\eps > 0$, we can find a tail $\sum_{k = N}^\infty c^{k - 1} d(p_1, p_2) < \eps$.
	So, for $ m > n > N$, we therefore get
	\[
		d(p_n, p_m) \le \sum_{k = n + 1}^\infty c^{k - 1} d(p_1, p_2) \le \sum_{k = N}^\infty c^{k - 1} d(p_1, p_2) < \eps,
	\]
	i.e. $\set{ p_n }$ is Cauchy.
	Since $(E, d)$ is complete, $p_n \to p \in E$.
	Moreover, since $T$ is Lipschitz, it's continuous.
	So,
	\[
		p = \lim T(p_n) = T(\lim p_n) = T(p),
	\]
	i.e. $p$ is indeed a fixed point.
\end{solution}
\end{problem}

\end{document}
